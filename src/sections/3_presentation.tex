\section{Fachgespräch/Präsentation}

\begin{taskitemwithoutcomment}{Anwesende Personen}
Beim Fachgespräch und der Präsentation nehmen einzig folgende Personen teil:
\begin{itemize}
    \item Kandidat
    \item Haupt- und Nebenexperten. Diese sind als Vertretung des Amts für Berufsbildung da für eine objektive Beurteilung und
          als Sicherstellung des Qualitätsnachweises.
    \item Verantwortliche Fachkraft (wenn gewünscht vom Kandidaten). Diese darf sich nur nach Aufforderung der Experten äussern.
    \item evtl. amtliche Beobachtungsperson
\end{itemize}
\end{taskitemwithoutcomment}

\begin{taskitem}{Wohlbefinden Kandidat}
Der Kandidat bestätigt, dass er sich gesundheitlich wohlfühlt und in der Lage ist, die Präsentation und das Fachgespräch durchzuführen.
Zudem bestätigt der Kandidat, dass es keine neuen Informationen gibt, welche die Bewertung beeinflussen. Sonst sollen diese hier notiert werden.
\end{taskitem}

\begin{taskitem}{Umfeld}
Für die Präsentation und das Fachgespräch steht ein störungsfreier Raum zur Verfügung. Zudem hat der Kandidat die Präsentationsunterlagen
im PkOrg hochgeladen.
\end{taskitem}

\begin{taskitem}{Durchführung der Präsentation und Demo}
Der Kandidat kann nun seine Präsentation starten und im Anschluss direkt die Demonstration seiner Arbeit durchführen. Die Präsentation
darf durch die Experten nicht unterbrochen werden aufgrund der zeitlichen Vorgaben. Unklarheiten sollen im Anschluss geklärt werden.
\end{taskitem}

\begin{taskitemwithoutcomment}{Durchführung des Fachgesprächs}
Im Anschluss zur Demonstration soll das vorbereitete Fachgespräch durchgeführt werden. Durch die Experten werden Fragen aus 6 Themenbereichen geprüft.
\end{taskitemwithoutcomment}

\begin{taskitem}{Notizen}
Falls weitere Punkte besprochen wurden, können diese hier notiert werden:
\end{taskitem}

\clearpage
