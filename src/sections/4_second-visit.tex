\section{Zweiter Besuch}
\begin{taskitem}{Vorbereitung für zweiten Besuch}
  Bereite einige Fragen vor. Sie sollen dir unter anderem ein Bild vermitteln, wie gross eine allfällige Fremdhilfe ist.
\end{taskitem}
\begin{taskitem}{Zeitplan, Journal und Dokumentation}
  Wie wird der \textbf{Zeitplan} eingehalten, wie wird das \textbf{Journal} geführt (tagesaktuell?). Schaue den Stand der \textbf{Dokumentation} an. Insbesondere die kontinuierlichen Kriterien, z.B. zur Tagesaktualität des Journals und Backups.
\end{taskitem}
\begin{taskitem}{Situation}
  Sind Probleme aufgetaucht? Kann das vorgegebene Ziel erreicht werden?
\end{taskitem}
\begin{taskitem}{Fachfragen}
  Verwickle den Kandidaten in ein ungezwungenes Fachgespräch. Lasse dir Teile der Arbeit beschreiben. Stelle Fragen zum fachlichen Inhalt der Aufgabe. Mache Notizen dazu. Denke dabei auch schon an mögliche Fragestellungen am Präsentationstermin und dem Fachgespräch.
\end{taskitem}
\begin{taskitem}{Erwartungen}
  Erkläre dem Kandidaten die Erwartungen an den IPA-Bericht: Gliederung, Umfang, Schwerpunkte und Inhalte. Vorgaben im Dokument \enquote{QV-Leitfaden.pdf} sind zu beachten. Firmennormen/Firmenvorlagen dürfen bzw. sollen angewendet werden. Erinnere auch daran, dass alle Programme und Scripts die während der Arbeit erstellt wurden, vollständig im Anhang des IPA-Berichts enthalten sein müssen.
\end{taskitem}
\begin{taskitem}{Präsentationstermin}
  Klären und erklären Sie die Organisation und den Ablauf der Präsentation und der obligatorischen Demo. Hat der Kandidat die richtige IPA-Durchführungsadresse auf PkOrg eingetragen bzw. für die Präsentation aktualisiert? Treffpunkt und Zugang unter allen Beteiligten (Nebenexperte, ...) absprechen.
\end{taskitem}
\begin{taskitem}{Abgabetermin}
  Mache den Kandidaten auf den Abgabetermin aufmerksam: IPA-Bericht: Hochladen der .pdf-Datei abgeschlossen am letzten Prüfungstag 18:00:00.000 Uhr bzw. 13:00:00.000 Uhr bei Halbtag (keine Toleranz). Verspätete Abgabe gibt Notenabzug. Es gilt die Systemzeit des PkOrg-Servers.
\end{taskitem}
\begin{taskitem}{Korrektur}
  Erinnere die verantwortliche Fachkraft daran, dass sie die Arbeit und den IPA-Bericht bis zur Präsentation unbedingt korrigieren, Positives und Negatives vermerken und bewerten muss.\\ Bespreche den Bewertungsbogen nochmals mit der verantwortlichen Fachkraft, um alle Missverständnisse auszuräumen. Die verantwortliche Fachkraft muss im PkOrg seinen Bewertungsvorschlag mit Begründung vor der Präsentation eintragen und bestätigen.
\end{taskitem}
\begin{taskitem}{Themenvorschläge}
  Bitte die verantwortliche Fachkraft, Themenvorschläge für das Fachgespräch frühzeitig zu melden. Das unterstützt eine gute Vorbereitung.
\end{taskitem}
