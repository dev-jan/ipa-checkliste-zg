\section{Bis zum Beginn}
Mit der Freigabe wirst du gegenüber dem Kandidaten und dem Lehrbetrieb zum verantwortlichen Experten. Der Kandidat muss sich nun als Schützling auf dich verlassen können! Achte darauf, dass du für den Kandidaten erreichbar bist.

\begin{taskitemwithoutcomment}{Kommunikation}
  Spreche in jeder Nachricht nur einen Fall an und formuliere dich verständlich und anständig, sodass auch Dritte sie lesen dürfen (z. B. bei der Akteneinsicht und bei Rekursen).
  Die Kommunikation sollte ausschliesslich über PkOrg stattfinden, damit die Nachvollziehbarkeit gewährleistet ist.
\end{taskitemwithoutcomment}
\begin{taskitemwithoutcomment}{Begrüssung}
  Begrüsse den Kandidaten und seine verantwortliche Fachkraft mit einem Eintrag in der History. Teile den Beteiligten mit, wie du erreichbar bist. Gebe die nächsten Schritte bekannt.
\end{taskitemwithoutcomment}
\begin{taskitem}{Besuchstermine}
  Vereinbaren Sie den 1. Termin mit der verantwortlichen Fachkraft und dem Kandidaten, am besten am 2. Tag der IPA. Vielleicht kannst du auch schon die weiteren Termine vereinbaren. Trage alle Besuchstermine in PkOrg ein. \textbf{Berücksichtige dabei auch die Verfügbarkeit vom NEX.}
\end{taskitem}