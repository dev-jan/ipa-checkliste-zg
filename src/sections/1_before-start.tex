\section{Vor der PA-Umsetzung}
Folgende Punkte sollen vor dem Start der PA Umsetzung durchgegangen werden. Es ist in der Verantwortung des Fachvorgesetzten,
diese Punkte mit dem Kandidaten zusammen anzuschauen.

\begin{taskitem}{Nachteilsausgleich}
Für den Prüfungskandidaten wurde kein Abkommen mit dem Berufsbildungsamt
getroffen, welches einen Nachteilausgleich zugesteht. Falls doch, ist das
Abkommen in den PkOrg-Dokumenten-Pool hochzuladen und in der
Aufgabenbeschreibung darauf hinzuweisen.
\end{taskitem}

\begin{taskitemwithoutcomment}{Sprache}
Es kommt einzig die Standard-Sprache Deutsch zum Einsatz.
Eine Abweichung davon erfordert im Grundsatz eine Vereinbarung mit dem
Berufsbildungsamt, diese ist in den PkOrg-Dokumenten-Pool hochzuladen und
muss entsprechend in der Aufgabenbeschreibung behandelt werden.
\end{taskitemwithoutcomment}

\begin{taskitemwithoutcomment}{Video-Checkliste}
Die verantwortliche Fachkraft sowie der Prüfungskandidat nehmen Kenntnis von
der Video-Konferenz-Checkliste (siehe PkOrg).
\end{taskitemwithoutcomment}

\begin{taskitemwithoutcomment}{Wichtige Punkte verantwortliche Fachkraft}
Die verantwortliche Fachkraft nimmt zur Kenntnis, dass...
\begin{itemize}
    \item sie während der PA-Umsetzung anwesend sein muss.
    \item sie den Projektauftrag verantwortet.
    \item der freigegebene Projektauftrag verbindlich ist.
    \item sie beim Zwischengespräch teilnehmen muss.
    \item die Experten dem Amtsgeheimnis und der Schweigepflicht unterstehen.
    \item rein konzeptionelle Arbeiten nicht möglich sind.
    \item für die schriftliche Kommunikation die entsprechende PkOrg-Funktion zu verwenden ist.
    \item PkOrg über eine "Timeout-Funktion" verfügt. Womit empfohlen wird,
          regelmässig zu speichern oder bspw. die Auftragsdefinition oder Bewertung
          vorgängig in einem separaten Dokument zu erfassen.
\end{itemize}
\end{taskitemwithoutcomment}

\begin{taskitemwithoutcomment}{Wichtige Punkte Prüfungskandidat}
Der Prüfungskandidat nimmt zur Kenntnis, dass...
\begin{itemize}
    \item der Bewertungsbogen und die Schulungsunterlagen zu studieren sind.
    \item der PA-Bericht als PDF hochzuladen ist.
    \item die hochgeladenen Dokumente bewertungsrelevant sind
    \item ein zu spätes Einreichen (Hochladen auf PkOrg inklusive Signatur; Vormittag 13:00:59h, Nachmittag 18:00:59h)
          der PA-Dokumentation zu einem Abzug von 0.5 Noten führt (Nulltoleranz).
    \item Dokumente, die mehr als 24 Stunden nach Terminablauf eingereicht werden, bewertungsirrelevant sind.
    \item die Präsentation sowohl in Hoch- als auch in Schweizerdeutsch erfolgen kann.
    \item der Soll-Zeitplan zum Ende des ersten PA-Tages in PkOrg hochgeladen
          werden muss (über die OnePage, History, neuer Eintrag, Hinweis
          Expertenteam/Kandidat/in). Sollte die PA am Nachmittag beginnen, so ist der
          Zeitplan bis zum nächsten Vormittag hochzuladen.
    \item Dokumentations-Vorlagen vorbereitet werden dürfen; selbsterklärend ohne
          Inhalt.
\end{itemize}
\end{taskitemwithoutcomment}

\begin{taskitemwithoutcomment}{Ungeplante Absenzen}
Der Prüfungskandidat und die verantwortliche Fachkraft wissen, wie bei
ungeplanten Absenzen seitens Kandidaten (grösser/gleich einem halben Tag) zu
verfahren ist.
\begin{itemize}
    \item Ungeplante Absenzen sind dem Hauptexperten unverzüglich zu melden.
    \item Bei Wiederaufnahme der Arbeit hinterlegt der Kandidat in PkOrg einen
          aktualisierten Zeitplan und informiert den Hauptexperten entsprechend.
    \item Ursache und Dauer des Unterbruchs werden durch den Kandidaten im
          Arbeitsjournal dokumentiert.
    \item Der Chefexperte wird auf Anweisung des Hauptexperten in PkOrg die
          Zeitplanung anpassen.
    \item Bezüglich Arztzeugnis gelten die firmeninternen Regeln.
\end{itemize}
\end{taskitemwithoutcomment}

\begin{taskitem}{Notizen vor dem Start}
Falls weitere Punkte besprochen wurden, können diese hier notiert werden:
\end{taskitem}

\clearpage
