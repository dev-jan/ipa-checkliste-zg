\section{Erster Besuch}
Denke daran, dass du bei deinen Besuchen die kantonale Prüfungskommission vertrittst. Korrektes Auftreten, Höflichkeit, Geduld und Pünktlichkeit setzen wir für diese Arbeit voraus. Dein Amt erfordert aber auch Vertraulichkeit und Verschwiegenheit. Für die verschiedenen Besuche und Besprechungen stehen Hilfsformulare zur Verfügung, welche benutzt werden sollten.

\begin{taskitem}{Vorstellungsrunde}
  Beginne das Gespräch mit einer Vorstellungsrunde. Gebe in erster Linie dem Kandidaten und der verantwortlichen Fachkraft Gelegenheit, sich vorzustellen. Beantworte aber auch die Fragen zu deiner Person und ggf. deinem Arbeitgeber.
\end{taskitem}
\begin{taskitem}{Rollen}
  Stelle die Rollen der beteiligten Personen vor. Erinnere die verantwortliche Fachkraft daran, dass du nun ihr Partner bist und der Kandidat sorgfältig bei der Arbeit beobachtet werden muss (Protokoll).
\end{taskitem}
\begin{taskitem}{Arbeitsplatz}
  Achte auf die Infrastruktur und das Umfeld. Ist ein ungestörtes Arbeiten möglich? Arbeitet der Kandidat an seinem üblichen Arbeitsplatz? Hat der Kandidat noch Nebenaufgaben zu erfüllen?
\end{taskitem}
\begin{taskitem}{Material und Vorarbeiten}
  Ist das Material vorhanden, bzw. sind die Voraussetzungen für die Durchführung gemäss Aufgabenstellung erfüllt? Sind die deklarierten Vorarbeiten erfolgreich abgeschlossen?
\end{taskitem}
\begin{taskitem}{Gemeinsames Verständnis der Aufgabe}
  Lasse dir die Aufgabenstellung in Form und Umfang nochmals bestätigen. Lese auch die Diskussion aus der Validierung nochmals durch. Hat es damals schon Fragen gegeben, die jetzt zu klären sind? Hat der Kandidat die Aufgabe verstanden?
\end{taskitem}
\begin{taskitem}{Bewertungskriterien}
  Bespreche die Bewertungskriterien. Ein gemeinsames Verständnis ist Vorbedingung für eine reibungslose Bewertung.
\end{taskitem}
\begin{taskitem}{Wichtige Dokumente}
  Haben die verantwortliche Fachkraft und der Kandidat die folgenden IPA-Dokumente gelesen?
  \begin{enumerate}
    \item Dokumente, welche beim Anmelden auf PkOrg bestätigt wurden
    \item QV-Leitfaden von \href{https://pk19.ch}{der Webseite der Prüfungskommission}
    \item QV-Termine von \href{https://pk19.ch}{der Webseite der Prüfungskommission}
  \end{enumerate}
  Diese Dokumente stellen im Kanton Zürich die aktuellen verbindlichen Vorgaben für den Inhalt und die Gestaltung des IPA-Berichts und können sich von anderen Kantonen unterscheiden.
\end{taskitem}
\begin{taskitem}{Fragen und Probleme}
  Erkläre, dass sich der Kandidat für alle Fragen und Eventualitäten (z.B. Probleme mit Hard- oder Software, aber auch Krankheit) an dich und nur an dich wenden muss (vgl. Abschnitt \enquote{Zwischenfälle: Pannen, Krankheit, Knapp, ...}).
\end{taskitem}
\begin{taskitem}{Zeitplan}
  Verlange den IPA-Zeitplan des Kandidaten. Idealerweise wird dieser beim Besuchstermin angeschaut. Der Kandidat soll den Zeitplan als .pdf-Datei in die History auf PkOrg laden. Falls dies nicht möglich ist, soll er den Zeitplan per E-Mail an dich senden und du lädst ihn nach dem Besuchstermin auf PkOrg hoch.
\end{taskitem}
\begin{taskitem}{Nächste Schritte}
  Bespreche die weiteren Termine. Der nächste Besuch sollte in der zweiten Hälfte der Facharbeit (Tag 7 oder 8) stattfinden. Die Präsentation findet idealerweise etwa eine Woche nach Abschluss der Facharbeit statt. Nehme immer auf die Bedürfnisse des Betriebes und des Nebenexperten Rücksicht. Findet sich kein geeigneter gemeinsamer Termin, soll der Nebenexperte die Arbeit abgeben und einem anderen Nebenexperten übergeben. Bitte die Termine immer auf PkOrg nachführen.
  Gib nochmal deine Erreichbarkeit bekannt und das üblicherweise die Kommunikation über PkOrg erfolgt.
\end{taskitem}
