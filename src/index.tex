% !TeX root = index.tex
% define document class
\documentclass{scrreprt}

% apply style package
\usepackage{./style}

% set variables
\newcommand{\varAuthor}{Jan Bucher}
\newcommand{\varCompany}{Amt für Berufsbildung Kanton Zug}
\newcommand{\varTitle}{Checkliste PA Informatik Kanton Zug}
\newcommand{\varVersion}{Version 2023.1}

% footer and headings
\rohead[\varTitle]{\varTitle}
\lofoot[\varCompany]{\varCompany}
\cfoot[\varAuthor\\Version \varVersion]{\varVersion}
\rofoot[Seite \pagemark{} von \pageref{LastPage}]{Seite \pagemark{} von \pageref{LastPage}}

% create document
\begin{document}
\begin{Form}
  \chapter*{\varTitle}

  \sbox\TBox{Titel der PA: }\TextField[charsize=10pt, width=\dimexpr\linewidth-\wd\TBox]{Arbeitstitel:}\\
  \sbox\TBox{Kandidat: }\TextField[charsize=10pt, width=\dimexpr\linewidth-\wd\TBox]{Kandidat:}\\
  \sbox\TBox{Lehrbetrieb: }\TextField[charsize=10pt, width=\dimexpr\linewidth-\wd\TBox]{Lehrbetrieb:}\\
  \sbox\TBox{Hauptexperte: }\TextField[charsize=10pt, width=\dimexpr\linewidth-\wd\TBox]{Hauptexperte:}\\

  \section{Vor der PA-Umsetzung}
Folgende Punkte sollen vor dem Start der PA Umsetzung durchgegangen werden. Es ist in der Verantwortung des Fachvorgesetzten,
diese Punkte mit dem Kandidaten zusammen anzuschauen.

\begin{taskitem}{Nachteilsausgleich}
Für den Prüfungskandidaten wurde kein Abkommen mit dem Berufsbildungsamt
getroffen, welches einen Nachteilausgleich zugesteht. Falls doch, ist das
Abkommen in den PkOrg-Dokumenten-Pool hochzuladen und in der
Aufgabenbeschreibung darauf hinzuweisen.
\end{taskitem}

\begin{taskitemwithoutcomment}{Sprache}
Es kommt einzig die Standard-Sprache Deutsch zum Einsatz.
Eine Abweichung davon erfordert im Grundsatz eine Vereinbarung mit dem
Berufsbildungsamt, diese ist in den PkOrg-Dokumenten-Pool hochzuladen und
muss entsprechend in der Aufgabenbeschreibung behandelt werden.
\end{taskitemwithoutcomment}

\begin{taskitemwithoutcomment}{Video-Checkliste}
Die verantwortliche Fachkraft sowie der Prüfungskandidat nehmen Kenntnis von
der Video-Konferenz-Checkliste (siehe PkOrg).
\end{taskitemwithoutcomment}

\begin{taskitemwithoutcomment}{Wichtige Punkte verantwortliche Fachkraft}
Die verantwortliche Fachkraft nimmt zur Kenntnis, dass...
\begin{itemize}
    \item sie während der PA-Umsetzung anwesend sein muss.
    \item sie den Projektauftrag verantwortet.
    \item der freigegebene Projektauftrag verbindlich ist.
    \item sie beim Zwischengespräch teilnehmen muss.
    \item die Experten dem Amtsgeheimnis und der Schweigepflicht unterstehen.
    \item rein konzeptionelle Arbeiten nicht möglich sind.
    \item für die schriftliche Kommunikation die entsprechende PkOrg-Funktion zu verwenden ist.
    \item PkOrg über eine "Timeout-Funktion" verfügt. Womit empfohlen wird,
          regelmässig zu speichern oder bspw. die Auftragsdefinition oder Bewertung
          vorgängig in einem separaten Dokument zu erfassen.
\end{itemize}
\end{taskitemwithoutcomment}

\begin{taskitemwithoutcomment}{Wichtige Punkte Prüfungskandidat}
Der Prüfungskandidat nimmt zur Kenntnis, dass...
\begin{itemize}
    \item der Bewertungsbogen und die Schulungsunterlagen zu studieren sind.
    \item der PA-Bericht als PDF hochzuladen ist.
    \item die hochgeladenen Dokumente bewertungsrelevant sind
    \item ein zu spätes Einreichen (Hochladen auf PkOrg inklusive Signatur; Vormittag 13:00:59h, Nachmittag 18:00:59h)
          der PA-Dokumentation zu einem Abzug von 0.5 Noten führt (Nulltoleranz).
    \item Dokumente, die mehr als 24 Stunden nach Terminablauf eingereicht werden, bewertungsirrelevant sind.
    \item die Präsentation sowohl in Hoch- als auch in Schweizerdeutsch erfolgen kann.
    \item der Soll-Zeitplan zum Ende des ersten PA-Tages in PkOrg hochgeladen
          werden muss (über die OnePage, History, neuer Eintrag, Hinweis
          Expertenteam/Kandidat/in). Sollte die PA am Nachmittag beginnen, so ist der
          Zeitplan bis zum nächsten Vormittag hochzuladen.
    \item Dokumentations-Vorlagen vorbereitet werden dürfen; selbsterklärend ohne
          Inhalt.
\end{itemize}
\end{taskitemwithoutcomment}

\begin{taskitemwithoutcomment}{Ungeplante Absenzen}
Der Prüfungskandidat und die verantwortliche Fachkraft wissen, wie bei
ungeplanten Absenzen seitens Kandidaten (grösser/gleich einem halben Tag) zu
verfahren ist.
\begin{itemize}
    \item Ungeplante Absenzen sind dem Hauptexperten unverzüglich zu melden.
    \item Bei Wiederaufnahme der Arbeit hinterlegt der Kandidat in PkOrg einen
          aktualisierten Zeitplan und informiert den Hauptexperten entsprechend.
    \item Ursache und Dauer des Unterbruchs werden durch den Kandidaten im
          Arbeitsjournal dokumentiert.
    \item Der Chefexperte wird auf Anweisung des Hauptexperten in PkOrg die
          Zeitplanung anpassen.
    \item Bezüglich Arztzeugnis gelten die firmeninternen Regeln.
\end{itemize}
\end{taskitemwithoutcomment}

\begin{taskitem}{Notizen vor dem Start}
Falls weitere Punkte besprochen wurden, können diese hier notiert werden:
\end{taskitem}

\clearpage

  \section{Zwischengespräch}
Folgende Punkte sollen während des Zwischengesprächs besprochen und kontrolliert werden vom Hauptexperten.

\begin{taskitemwithoutcomment}{Anwesenheit}
Beim Zwischengespräch sollte der Prüfungskandidat, der Fachvorgesetzte und der Hauptexperte anwesend sein.
\end{taskitemwithoutcomment}

\begin{taskitem}{Zeitplan}
Der Soll-Zeitplan der PA wurde vom Kandidaten im PkOrg hochgeladen und soll während des Zwischengespräches besprochen werden.
Der Kandidat und der Fachvorgesetzte sind immer noch der Überzeugung, dass die PA in den vorgegebenen 80 Stunden
geleistet werden kann.
\end{taskitem}

\begin{taskitem}{Zielsetzung}
Die Zielsetzung der Arbeit hat sich nicht geändert und bleibt gleich wie im PkOrg angegeben.
\end{taskitem}

\begin{taskitemwithoutcomment}{Bewertung der Arbeit}
Die verantwortliche Fachkraft weiss, wie die Arbeit zu bewerten ist.
Sie bewertet die Arbeit in PkOrg und schliesst diesen Schritt einige Tage vor dem
Fachgespräch ab. Nach Möglichkeit stellt die verantwortliche Fachkraft den
Experten ergänzend eine Zusammenfassung der Eindrücke zur Verfügung.\\
Für das Fachgespräch muss ein störungsfreier Raum zur Verfügung gestellt werden vom Lehrbetrieb (z.B. Sitzungszimmer).
Für die anschliessende Bewertung zusammen mit dem Fachvorgesetzten (ohne Kandidat) soll der Raum auch zur Verfügung stehen.
Zudem soll ein Wireless-Zugang für die Experten bereitgestellt werden, sodass die Bewertung gleich vor Ort im PkOrg gemacht
werden kann.
\end{taskitemwithoutcomment}

\begin{taskitemwithoutcomment}{Informationen für den Kandidaten}
Dem Kandidaten wurde Folgendes mitgeteilt:
\begin{itemize}
    \item Der Kandidat muss vor dem Fachgespräch die Präsentationsunterladen auf PkOrg hochladen.
    \item Der Kandidat hat den Prüfungsexperten zum Zeitpunkt des Fachgesprächs je ein Exemplar der Präsentationsunterlagen abzugeben.
    \item Der Kandidat entscheidet beim Fachgespräch, ob der Fachvorgesetzte auch anwesend ist.
\end{itemize}
\end{taskitemwithoutcomment}

\begin{taskitemwithoutcomment}{Fragen beantworten}
Falls der Kandidat noch allfällige Fragen hat, sollen diese durch den Hauptexperten beantwortet werden.
\end{taskitemwithoutcomment}

\begin{taskitem}{Notizen Zwischengespräch}
Falls weitere Punkte besprochen wurden, können diese hier notiert werden:
\end{taskitem}

\clearpage

  \section{Fachgespräch/Präsentation}

\begin{taskitemwithoutcomment}{Anwesende Personen}
Beim Fachgespräch und der Präsentation nehmen einzig folgende Personen teil:
\begin{itemize}
    \item Kandidat
    \item Haupt- und Nebenexperten. Diese sind als Vertretung des Amts für Berufsbildung da für eine objektive Beurteilung und
          als Sicherstellung des Qualitätsnachweises.
    \item Verantwortliche Fachkraft (wenn gewünscht vom Kandidaten). Diese darf sich nur nach Aufforderung der Experten äussern.
    \item evtl. amtliche Beobachtungsperson
\end{itemize}
\end{taskitemwithoutcomment}

\begin{taskitem}{Wohlbefinden Kandidat}
Der Kandidat bestätigt, dass er sich gesundheitlich wohlfühlt und in der Lage ist, die Präsentation und das Fachgespräch durchzuführen.
Zudem bestätigt der Kandidat, dass es keine neuen Informationen gibt, welche die Bewertung beeinflussen. Sonst sollen diese hier notiert werden.
\end{taskitem}

\begin{taskitem}{Umfeld}
Für die Präsentation und das Fachgespräch steht ein störungsfreier Raum zur Verfügung. Zudem hat der Kandidat die Präsentationsunterlagen
im PkOrg hochgeladen.
\end{taskitem}

\begin{taskitem}{Durchführung der Präsentation und Demo}
Der Kandidat kann nun seine Präsentation starten und im Anschluss direkt die Demonstration seiner Arbeit durchführen. Die Präsentation
darf durch die Experten nicht unterbrochen werden aufgrund der zeitlichen Vorgaben. Unklarheiten sollen im Anschluss geklärt werden.
\end{taskitem}

\begin{taskitemwithoutcomment}{Durchführung des Fachgesprächs}
Im Anschluss zur Demonstration soll das vorbereitete Fachgespräch durchgeführt werden. Durch die Experten werden Fragen aus 6 Themenbereichen geprüft.
\end{taskitemwithoutcomment}

\begin{taskitem}{Notizen}
Falls weitere Punkte besprochen wurden, können diese hier notiert werden:
\end{taskitem}

\clearpage

  \section{Nach dem Fachgespräch}
Nach dem Fachgespräch verlässt der Kandidat den Raum und die Experten besprechen die Arbeit zusammen mit dem Fachvorgesetzten.

\begin{taskitem}{Bewertung Teil C}
Direkt im Anschluss des Fachgespräches soll dieses im PkOrg bewertet werden.
\end{taskitem}

\begin{taskitem}{Besprechen der gesamten Bewertung}
Die gesamte Bewertung der Experten wird mit der des Fachvorgesetzten verglichen und besprochen. Ziel sollte es sein, dass sich alle
Parteien einig sind. Sollte dies nicht möglich sein, so ist der Chefexperte entsprechend zu informieren.

Die Bewertung wird anschliessend vom Hauptexperten, dem Nebenexperten und dem Fachvorgesetzten signiert im PkOrg.
\end{taskitem}

\begin{taskitemwithoutcomment}{Hochladen der Dokumente}
Alle Dokumente, welche zur Bewertung relevant sind, sollen im PkOrg hochgeladen werden. Diese sollen von der Sichtbarkeit so
eingestellt werden, dass diese dem Kandidaten nicht sichtbar sind. Am Schluss soll diese Checkliste im PkOrg ebenfalls abgelegt werden.
\end{taskitemwithoutcomment}

\begin{taskitem}{Notizen}
Falls es noch weitere relevanten Punkte gibt, können diese hier notiert werden:
\end{taskitem}

\end{Form}
\end{document}
