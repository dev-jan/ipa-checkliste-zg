\section{Abschluss}
Die Prüfungsleitung ist dir dankbar, wenn du die folgenden Arbeiten umgehend erledigst.

\begin{taskitemwithoutcomment}{Abgleich der Punkte und Hinweise}
  Punktzahlen und Hinweise zur IPA und zum Lehr-/Praktikumsbetrieb sofort in PkOrg eintragen, sofern du die Punkte nicht schon online eingetragen hast. Bei verspäteter Abgabe zusätzlich E-Mail an den Chefexperten und Eintrag ins Feld \enquote{Expertenbericht} auf PkOrg, welcher die Umstände beschreibt. Der bei zu spätem Upload automatisch gesetzte Haken für einen Abzug darf nicht gelöscht werden.
\end{taskitemwithoutcomment}
\begin{taskitemwithoutcomment}{Uneinigkeit}
  Falls grosse Uneinigkeiten (> 0.3 Notenpunkte oder genügend/ungenügend) bei der Bewertung aufgetreten sind, fordere beim Chefexperten eine Zweitbeurteilung an. (Ausnahmefälle)
\end{taskitemwithoutcomment}
\begin{taskitemwithoutcomment}{Digitalisierung}
  Scanne alle IPA Unterlagen ein und laden diese unter PkOrg bei den Dokumenten hoch, sodass diese für die Prüfungsleitung ersichtlich sind.
\end{taskitemwithoutcomment}
\begin{taskitemwithoutcomment}{Rückfragen}
  Am Tag der Notenklausur solltest du für Rückfragen erreichbar sein. Trage den Termin in deine Agenda ein.
\end{taskitemwithoutcomment}
\begin{taskitemwithoutcomment}{Schweigepflicht}
  Denke an deine Schweigepflicht: Ausserhalb des Bewertungsverfahrens darf weder über den Inhalt der IPA noch über die Bewertung gesprochen werden. Die Rechte am Resultat der IPA gehören der Lehr-/Praktikumsfirma.
\end{taskitemwithoutcomment}

\textbf{Abschliessend gebührt Ihnen ein ganz herzliches Dankeschön!}
Wir sind uns bewusst, dass es nicht selbstverständlich ist, dass Sie die Arbeit des Prüfungsexperten übernehmen. Die Berufslehre geniesst nicht zuletzt wegen des Prüfungswesens ein hohes Ansehen - schön, dass du diese Verantwortung mitträgst!