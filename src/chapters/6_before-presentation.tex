\chapter{Bis zur Präsentation}
Nehme dir Zeit, den IPA-Bericht gründlich durchzulesen. Denke daran, das ist ein Bericht einer ausgebildeten Fachkraft!

\begin{taskitemwithoutcomment}{Bericht korrigieren}
  Den Teil A und B anhand des abgegebenen Berichts, sowie den Eindrücken bei den Besuchsterminen bewerten und auf PkOrg eintragen. Lese die Bewertung der verantwortlichen Fachkraft und schaue, ob ihr derselben Meinung seid.
\end{taskitemwithoutcomment}
\begin{taskitemwithoutcomment}{Fachgespräch vorbereiten}
  Acht bis zehn Gesprächsthemen für das Fachgespräch vorbereiten. Worauf schaust du bei der Antwort? Was erwartest du? Benutze das Vorbereitungsformular, welches auf PkOrg abgelegt ist.\\ Nehme schon jetzt Themen der verantwortlichen Fachkraft auf. So kannst du ein ausgewogenes Fachgespräch vorbereiten. Das Fachgespräch darf nicht mit einer Berufskundeprüfung verwechselt werden. Natürlich ist auch Wissen gefragt, aber immer im Zusammenhang mit der Facharbeit. Das Fachgespräch soll aufzeigen, ob der Kandidat in seiner Vertiefungsrichtung kompetent Auskunft geben kann.
\end{taskitemwithoutcomment}
\begin{taskitem}{Fachthemen aneignen}
  Informiere dich notfalls zusätzlich zum Thema. Besonders dann, wenn du mit der verantwortlichen Fachkraft oder dem Kandidaten an den Besuchsterminen zu wenig \enquote{fachsimpeln} konnten.
\end{taskitem}
