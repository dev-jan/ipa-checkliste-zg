\chapter{Vorbereitungsphase}
In dieser Phase wird die Aufgabenbeschreibung von den Validationsexperten betreut. Kontakte zu den Kandidaten, verantwortlichen Fachkräften und Berufsbildner erfolgen in der Regel über die Validationsexperten oder durch die Prüfungsleitung.

\begin{taskitemwithoutcomment}{Auswahl}
  \textbf{Wähle} die Arbeit sorgfältig \textbf{aus}. Es ist optimal und wird von der verantwortlichen Fachkraft geschätzt, wenn die Experten im Thema der Arbeit sind. Zudem sind die \textbf{Ausstandsregeln} zu beachten.
\end{taskitemwithoutcomment}
\begin{taskitem}{Validierung}
  Beteilige dich an der Validierung. Lese die Aufgabenstellung sorgfältig durch und überlege, ob du den Erfüllungsgrad und die Qualität der Facharbeit beurteilen kannst. Der Validationsexperte ist dankbar für deine Hinweise im PkOrg-Validationsdialog (nicht History). Beachte die Netiquette!
\end{taskitem}
\begin{taskitem}{Verfügbarkeit}
  Überprüfe die Termine, welche in der Beschreibung der Facharbeit angegeben wurden. Bist du eventuell in dieser Zeit abwesend? Organisiere allenfalls einen Abtausch (bitte Arbeiten nicht einfach nur abgeben) oder eine Verschiebung der Arbeit. Prüfe, ob die Funktion des Berufsbildners und der verantwortlichen Fachkraft richtig eingetragen sind.
\end{taskitem}