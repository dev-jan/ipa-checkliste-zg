\chapter{Präsentation, Demonstration und Fachgespräch}
Beachte, dass dieser Tag gerade für den Kandidaten ein spezieller Tag ist! Vermeide Verspätungen! Verleihe dem Anlass ruhig etwas Würdevolles! An der Präsentation nehmen nur der Kandidat, die verantwortliche Fachkraft, die Experten und ev. der Berufsbildner (bei der Präsentation und Demonstration) teil. Maximal eine weitere Person, welche massgeblich an der Umsetzung der Facharbeit beteiligt war und deren Fachkenntnis bei der Bewertung Wertvolles beitragen kann, darf bei der Präsentation auf Antrag bei der Prüfungsleitung ebenfalls teilnehmen.

\begin{taskitem}{Begrüssung}
  Du eröffnest die Präsentation als Teil der offiziellen Abschlussprüfung. Stelle weitere Teilnehmende vor! Frage den Kandidaten, ob er gesund sei und in der Lage, die Prüfung zu absolvieren. Falls nicht, muss der Kandidat zum Arzt und sich ein ärztliches Zeugnis ausstellen lassen. In diesem Fall wird der Termin verschoben und es muss ein neuer Termin vereinbart werden.
\end{taskitem}
\begin{taskitemwithoutcomment}{Wortübergabe}
  Übergebe dem Kandidaten das Wort. Wünsche ihm viel Erfolg.
\end{taskitemwithoutcomment}
\begin{taskitem}{Sprache}
  Der Vortrag muss vollständig in Hochdeutsch gehalten werden.
\end{taskitem}
\begin{taskitemwithoutcomment}{Zwischenfragen}
  Stelle keine Fragen während des Vortrags, auch wenn der Kandidat dir das zugestehen würde. Dazu ist später noch Zeit.
\end{taskitemwithoutcomment}
\begin{taskitem}{Präsentation}
  Achte auf den Aufbau und den Inhalt des Vortrages, aber auch auf die Vortragsdauer, den Einsatz der vorhandenen Mittel und den Vortragsstil. Die Kandidaten haben in der Schule gelernt, wie man einen Vortrag hält.
\end{taskitem}
\begin{taskitem}{Demonstration}
  Während der anschliessenden obligatorischen Demo darfst du spontan Fragen stellen. Diese sollten aber den vorbereiteten Ablauf nicht stören. Notfalls können Sie noch einmal darauf zurückkommen. Mundart ist erlaubt.
\end{taskitem}
\begin{taskitemwithoutcomment}{Übergang zum Fachgespräch}
  Nach dem Abschluss des Vortrages und der Demo übernimmst du wieder die Führung für das Fachgespräch. Mundart ist erlaubt.
\end{taskitemwithoutcomment}
\begin{taskitemwithoutcomment}{Durchführung des Fachgesprächs}
  Du als Hauptexperte führst das Gespräch. Lasse spontane Fragen der verantwortlichen Fachkraft und des Nebenexperten zu. Achte aber auf die Zeit: pro Themenbereich sind maximal 10 Minuten vorgesehen. Achte darauf, dass ausschliesslich der Kandidat die Fragen beantwortet. Die verantwortliche Fachkraft darf Präzisierungen erst bei der Notengebung abgeben. Notiere (oder besser der Nebenexperte) alle Fragen sowie in Stichworten die Antworten bzw. deren Qualität (das Formular mit Ihrer Vorbereitung hilft dabei). Das Fachgespräch dauert mindestens 30 und maximal 60 Minuten.
\end{taskitemwithoutcomment}